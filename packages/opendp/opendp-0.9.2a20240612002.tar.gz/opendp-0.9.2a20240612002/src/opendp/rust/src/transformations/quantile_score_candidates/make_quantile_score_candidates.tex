\documentclass{article}
% common styling and macros shared by all proof files

\usepackage[top=1in, right=1in, left=1in, bottom=1.5in]{geometry}

\usepackage{amsmath,amsthm,amsfonts,amssymb,amscd}
\usepackage{listings}
\usepackage{hyperref}
\usepackage{xcolor}
\usepackage{xr}

\usepackage{enumerate} 
\usepackage{physics}
\usepackage{fancyhdr}
\usepackage{hyperref}
\usepackage{graphicx}
\usepackage{tcolorbox}
\usepackage{catchfile}
\usepackage{pdftexcmds}
\usepackage[T1]{fontenc}

% hyperref
\hypersetup{
  colorlinks=true,
  linkcolor=blue,
  linkbordercolor={0 0 1}
}

% \contrib macro to indicate inclusion in "contrib".
\usepackage{tcolorbox}
\newtcolorbox{warn_box}{colback=red!5!white,colframe=red!75!black}
\newcommand{\contrib}{{\begin{warn_box}This proof resides in \textbf{``contrib''} because it has not completed the vetting process.\end{warn_box}}} 
\newcommand{\floatingPoint}{{\begin{warn_box}This implementation is susceptible to floating-point vulnerabilities.\end{warn_box}}} 

% asOfCommit macro to version a code dependency. Arguments:
%    #1: relative path to file you are dependent on
%    #2: commit hash it was last edited. If outdated, this should be the second hash in the footnoote. Otherwise,
%            git log -n 1 --pretty=format:%h -- path/to/file.rs
\makeatletter
\ifnum\pdf@shellescape=1
   % "private" command that builds a link to a blob
  \newcommand{\linkOpendpBlob}[3]{%
    \href{https://github.com/opendp/opendp/blob/#1/#2#3}{\path{#3} at commit #1}}

  % latex macro expansion has a separate phase for \input evaluation
  %     immediately evaluate a command to write a temp file to ./out containing the current directory
  \immediate\write18{[ ! -f out/cwd.txt ] && (mkdir -p out && git rev-parse --show-prefix | sed "s|_|\@backslashchar\@backslashchar\@backslashchar_|g" > out/cwd.txt)}
  %     ...and then retrieve the current working directory by loading the temp file
  \CatchFileDef\GitWorkingDir{out/cwd.txt}{\endlinechar=-1}

  % command for building the (up to date) or (outdated) status
  \newcommand{\fileStatus}[2]{%
  \setbox0=\hbox{\input|"git --no-pager log -n1 --pretty='\@percentchar H' #1 | grep -E '^#2.*'"\unskip}\ifdim\wd0=0pt
        (outdated\footnote{See new changes with \texttt{git diff #2..\input|"git --no-pager log -n1 --pretty='\@percentchar h' #1" \GitWorkingDir\path{#1}}})\else
        (up to date)\fi
  }

  \newcommand{\asOfCommit}[2]{%
      % permalink the target
      \linkOpendpBlob{#2}{\GitWorkingDir}{#1}
      % conditionally add (outdated) or (up to date) depending on matching commit hash
      \fileStatus{#1}{#2}%
  }
\else
  % simplified command if shell-escape not enabled
  \newcommand{\asOfCommit}[2]{#1 at commit #2 (unknown status\footnote{Shell-escape is not enabled. Enable \texttt{--shell-escape} to check if this proof is up-to-date with the code.})}
\fi
\makeatother

% \vettingPR macro to link a PR. Arguments:
%    #1: PR number
\newcommand{\vettingPR}[1]{\href{https://github.com/opendp/opendp/pull/#1}{Pull Request \##1}}

% for links to rustdoc items in OpenDP. Arguments:
%    #1: path to item, and designation as trait, struct, fn, etc.
%    #2: item name
\makeatletter
\ifnum\pdf@shellescape=1
  % latex macro expansion has a separate phase for \input evaluation
  %     immediately evaluate a command to write a temp file to ./out containing the base path
  \immediate\write18{[ ! -f out/rustdoc.txt ] && mkdir -p out && ([ -z `kpsewhich --var-value OPENDP_RUSTDOC_PORT` ] && echo "https://docs.rs/opendp/`head -n 1 \@backslashchar`git rev-parse --show-toplevel\@backslashchar`/VERSION | sed 's|.*-dev.*|latest|g'`" || echo "http://localhost:`kpsewhich --var-value OPENDP_RUSTDOC_PORT`") > out/rustdoc.txt}
  %     ...and then retrieve the base path by loading the temp file
  \CatchFileDef\OpenDPRustdocBase{out/rustdoc.txt}{\endlinechar=-1}
\else
  % if shell commands are not enabled, just claim latest
  \newcommand{\OpenDPRustdocBase}{https://docs.rs/opendp/latest}
\fi
\makeatother
\newcommand{\rustdoc}[2]{\href{\OpenDPRustdocBase/opendp/#1.#2.html}{\texttt{#2}}}

% for links to external dependencies. Arguments:
%    #1: crate name
%    #2: path to item, and designation as trait, struct, fn, etc.
%    #3: item name
\newcommand{\docsrs}[3]{\href{https://docs.rs/#1/latest/#1/#2.#3.html}{\texttt{#3}}}

% minted (pseudocode)
\definecolor{codegreen}{rgb}{0,0.6,0}
\definecolor{codegray}{rgb}{0.5,0.5,0.5}
\definecolor{codepurple}{rgb}{0.58,0,0.82}
\definecolor{backcolour}{rgb}{0.95,0.95,0.92}

\lstdefinestyle{mystyle}{
    backgroundcolor=\color{backcolour},   
    commentstyle=\color{codegreen},
    keywordstyle=\color{magenta},
    numberstyle=\tiny\color{codegray},
    stringstyle=\color{codepurple},
    basicstyle=\ttfamily\footnotesize,
    breakatwhitespace=false,         
    breaklines=true,                 
    captionpos=b,                    
    keepspaces=true,                 
    numbers=left,                    
    numbersep=5pt,                  
    showspaces=false,                
    showstringspaces=false,
    showtabs=false,                  
    tabsize=2
}

\lstset{style=mystyle}

% common commands
\theoremstyle{definition}
\newtheorem{theorem}{Theorem}[section]
\newtheorem{lemma}[theorem]{Lemma}
\newtheorem{definition}[theorem]{Definition}
\newtheorem{warning}{Warning}
\newtheorem{corollary}{Corollary}
\newtheorem{proposition}{Proposition}
\newtheorem{remark}{Remark}
\newtheorem{observation}{Observation}
\newtheorem{note}{Note}

\newcommand{\vicki}[1]{{ {\color{olive}{(vicki)~#1}}}}
\newcommand{\hanwen}[1]{{ {\color{purple}{(hanwen)~#1}}}}
\newcommand{\zach}[1]{{ {\color{red}{(zach)~#1}}}}

\newcommand{\MultiSet}{\mathrm{MultiSet}}
\newcommand{\len}{\mathrm{len}}
\newcommand{\din}{\texttt{d\_in}}
\newcommand{\dout}{\texttt{d\_out}}
\newcommand{\T}{\texttt{T} }
\newcommand{\F}{\texttt{F} }
\newcommand{\Map}{\texttt{Map}}
\newcommand{\X}{\mathcal{X}}
\newcommand{\Y}{\mathcal{Y}}
\newcommand{\True}{\texttt{True}}
\newcommand{\False}{\texttt{False}}
\newcommand{\clamp}{\texttt{clamp}}
\newcommand{\function}{\texttt{function}}
\newcommand{\float}{\texttt{float }}
\newcommand{\questionc}[1]{\textcolor{red}{\textbf{Question:} #1}}


\newcommand{\validTransformation}[2]{%
  For every setting of the input parameters #1 to #2 such that the given preconditions
  hold, #2 raises an exception (at compile time or run time) or returns a valid transformation. A valid transformation has the following properties:
  \begin{enumerate}
      \item \textup{(Appropriate output domain).} 
      For every element $v$ in \texttt{input\_domain}, $\function(v)$ is in \texttt{output\_domain} or raises a data-independent runtime exception.
      
      \item \textup{(Stability guarantee).} 
      For every pair of elements $u, v$ in \texttt{input\_domain} and for every pair $(\din, \dout)$, 
      where \din\ has the associated type for \texttt{input\_metric} and \dout\ has the associated type for \\ \texttt{output\_metric}, 
      if $u, v$ are \din-close under \texttt{input\_metric}, $\texttt{stability\_map}(\din)$ does not raise an exception,
      and $\texttt{stability\_map}(\din) \leq \dout$, 
      then $\function(u), \function(v)$ are $\dout$-close under \texttt{output\_metric}.
  \end{enumerate}
}


\newcommand{\validMeasurement}[2]{%
  For every setting of the input parameters #1 to #2 such that the given preconditions
  hold, #2 raises an exception (at compile time or run time) or returns a valid measurement. A valid measurement has the following property:
  \begin{enumerate}
      \item \textup{(Privacy guarantee).}
      For every pair of elements $u, v$ in \texttt{input\_domain} and for every pair $(\din, \dout)$,
      where \din\ has the associated type for \texttt{input\_metric} and \dout\ has the associated type for \\ \texttt{output\_measure},
      if $u, v$ are \din-close under \texttt{input\_metric}, $\texttt{privacy\_map}(\din)$ does not raise an exception,
      and $\texttt{privacy\_map}(\din) \leq \dout$,
      then $\function(u), \function(v)$ are $\dout$-close under \texttt{output\_measure}.
  \end{enumerate}
}


% \externaldocument[trans_mod:]{../mod}
% \ref{trans_mod:thrm:privacy-proof}

\title{\texttt{fn make\_quantile\_score\_candidates}}
\author{Michael Shoemate \and Christian Covington \and Ira Globus-Harrus}
\begin{document}
\maketitle  


\contrib

Proves soundness of \rustdoc{transformations/fn}{make\_quantile\_score\_candidates} 
in \asOfCommit{mod.rs}{f5bb719}.
\texttt{make\_quantile\_score\_candidates} returns a Transformation that 
takes a numeric vector database and a vector of numeric quantile candidates,
and returns a vector of scores, where higher scores correspond to more accurate candidates.

\subsection*{Vetting History}
\begin{itemize}
    \item \vettingPR{456}
\end{itemize}

\section{Intuition}
The quantile score function scores each $c$ in a set of candidates $C$.

\begin{equation}
\begin{array}{rl}
    s_i &= -|(1 - \alpha) \cdot \#(X < C_i) - \alpha \cdot \#(X > C_i)|
\end{array}
\end{equation}

Where $\#(X < C_i) = |\{x \in X | x < C_i\}|$ is the number of values in $X$ less than $C_i$, 
and similarly for other variations of inequalities.
The scalar score function can be equivalently stated:
\begin{align}
    s_i &= -|(1 - \alpha) \cdot \#(X < c) - \alpha \cdot \#(X > c)| \\
    &= -|(1 - \alpha) \cdot \#(X < c) - \alpha \cdot (|X| - \#(X < c) - \#(X = c))| \\
    &= -|\#(X < c) - \alpha \cdot (|X| - \#(X = c))|
\end{align}

It has an intuitive interpretation as $-|candidate\_rank - ideal\_rank|$, 
where the absolute distance between the candidate and ideal rank penalizes the score.
The ideal rank does not include values in the dataset equal to the candidate.
This scoring function considers higher scores better, 
and the score is maximized at zero when the candidate rank is equivalent to the rank at the ideal $\alpha$-quantile.

The scalar scorer is almost equivalent to Smith's\cite{Smith11}, but adjusts for a source of bias when there are values in the dataset equal to the candidate.
For comparison, we can equivalently write the OpenDP scorer as if there were some $\alpha$-discount on dataset entries equal to the candidate.

\[
\begin{array}{cl}
    OpenDP &-|\#(X < c) + \alpha \cdot \#(X = c) - \alpha \cdot |X|| \\
    Smith &-|\#(X < c) + 1 \cdot \#(X = c) - \alpha \cdot |X||
\end{array}
\]

Observing that $\#(X \leq c) = \#(X < c) + 1 \cdot \#(X = c)$.


\subsection{Examples}

Let $X = \{0,1,2,3,4\}$ and $\alpha = 0.5$ (median):

\begin{align*}
    score(X, 0, \alpha) = -|0 - .5 \cdot (5 - 1)| = -2 \\
    score(X, 1, \alpha) = -|1 - .5 \cdot (5 - 1)| = -1 \\
    score(X, 2, \alpha) = -|2 - .5 \cdot (5 - 1)| = -0 \\
    score(X, 3, \alpha) = -|3 - .5 \cdot (5 - 1)| = -1 \\
    score(X, 4, \alpha) = -|4 - .5 \cdot (5 - 1)| = -2
\end{align*}

The score is maximized by the candidate at the true median.

Let $X = \{0,1,2,3,4,5\}$ and $\alpha = 0.5$ (median):

\begin{align*}
    score(X, 0, \alpha) = -|0 - .5 \cdot (6 - 1)| = -2.5 \\
    score(X, 1, \alpha) = -|1 - .5 \cdot (6 - 1)| = -1.5 \\
    score(X, 2, \alpha) = -|2 - .5 \cdot (6 - 1)| = -0.5 \\
    score(X, 3, \alpha) = -|3 - .5 \cdot (6 - 1)| = -0.5 \\
    score(X, 4, \alpha) = -|4 - .5 \cdot (6 - 1)| = -1.5 \\
    score(X, 5, \alpha) = -|5 - .5 \cdot (6 - 1)| = -2.5
\end{align*}

The two candidates nearest the median are scored equally and highest.

Let $X = \{0,1,2,3,4\}$ and $\alpha = 0.25$ (first quartile):

\begin{align*}
    score(X, 0, \alpha) = -|0 - .25 \cdot (5 - 1)| = -1 \\
    score(X, 1, \alpha) = -|1 - .25 \cdot (5 - 1)| = -0 \\
    score(X, 2, \alpha) = -|2 - .25 \cdot (5 - 1)| = -1 \\
    score(X, 3, \alpha) = -|3 - .25 \cdot (5 - 1)| = -2 \\
    score(X, 4, \alpha) = -|4 - .25 \cdot (5 - 1)| = -3
\end{align*}

As expected, the score is maximized when $c = 1$.

Let $X = \{0,1,2,3,4,5\}$ and $\alpha = 0.25$ (first quartile):

\begin{align*}
    score(X, 0, \alpha) = -|0 - .25 \cdot (6 - 1)| = -1.25 \\
    score(X, 1, \alpha) = -|1 - .25 \cdot (6 - 1)| = -0.25 \\
    score(X, 2, \alpha) = -|2 - .25 \cdot (6 - 1)| = -0.75 \\
    score(X, 3, \alpha) = -|3 - .25 \cdot (6 - 1)| = -1.75 \\
    score(X, 4, \alpha) = -|4 - .25 \cdot (6 - 1)| = -2.75 \\
    score(X, 5, \alpha) = -|5 - .25 \cdot (6 - 1)| = -3.75
\end{align*}

The ideal rank is 1.25. The nearest candidate, 1, has the greatest score, followed by 2, and then 0. 


\section{Finite Data Types}
The previous equation assumes the existence of real numbers to represent $\alpha$.
We instead assume $\alpha$ is rational, such that $\alpha = \frac{\alpha_{num}}{\alpha_{den}}$.
Multiply the equation through by $\alpha_{den}$ to get the following, 
which only uses integers:

\begin{equation}
    \textrm{score}(X, c, \alpha_{num}, \alpha_{den}) = -|\alpha_{den} \cdot \#(X < c) - \alpha_{num} \cdot (|X| - \#(X = c))|
\end{equation}

This adjustment also increases the sensitivity by a factor $\alpha_{den}$, 
but does not affect the utility.
We now make the scoring strictly non-negative.
\begin{itemize}
    \item Drop the negation and instead configure the exponential mechanism to minimize the score.
    \item Compute the absolute difference in a function that swaps the order of arguments to keep the sign positive.
\end{itemize}

\begin{equation}
    \textrm{score}(X, c, \alpha_{num}, \alpha_{den}) = \mathrm{abs\_diff}(\alpha_{den} \cdot \#(X < c), \alpha_{num} \cdot (|X| - \#(X = c)))
\end{equation}

To prevent a numerical overflow when computing the arguments to \texttt{abs\_diff}, 
first choose a data type that the scores are to be represented in.
If the number of records is greater than can be represented in this data type, 
then sample the dataset down to at most this number of records.
Notice that when any given record is added or removed, 
the counts differ by no more than they would have without this sampling down.
In the OpenDP implementation, the dataset size may be no greater than the max value of a Rust usize, 
because each index into the dataset maps to a distinct computer memory address.

Now allocate some of the bits of the data type for the alpha denominator,
and use the remaining bits for counts of up to $l$, where $l$ is the effective dataset size.
From this set-up, we choose an $\alpha_{den}$ such that $\alpha_{den} \cdot l$ is representable.
Since $\alpha_{num} \le \alpha_{den}$, $\alpha_{num} \cdot l$ is representable.
Since the dataset size fits in the choice of data type, then $|X|$ is representable.
Therefore, no quantity in the following equation is not representable.

\begin{equation}
    \textrm{score}(X, c, \alpha_{num}, \alpha_{den}, l) = \mathrm{abs\_diff}(\alpha_{den} \cdot \min(\#(X < c), l), \alpha_{num} \cdot \min(|X| - \#(X = c), l))
\end{equation}

Should we compute counts with a 64-bit integer, we might choose $\alpha_{den}$ to be 10,000.
This would allow for a fine fractional approximation of alpha,
while still leaving enough room for datasets on the order of $10^{15}$ elements.

\section{Hoare Triple}
\subsection*{Precondition}
\begin{itemize}
    \item \texttt{TIA} (input atom type) is a type with trait \rustdoc{traits/trait}{Number}.
    \item \texttt{A} (alpha type) is a type with trait \rustdoc{traits/trait}{Float}.
    \item \texttt{MI} is a type with trait \rustdoc{transformations/trait}{ARDatasetMetric}.
\end{itemize}

\subsection*{Function}
\label{sec:python-pseudocode}
\lstinputlisting[language=Python,firstline=2]{./pseudocode/make_quantile_score_candidates.py}


\subsection*{Postcondition}
\validTransformation
    {(\texttt{input\_domain, input\_metric, candidates, alpha, TIA, A, MI})}
    {\texttt{make\_quantile\_ score\_candidates}}

\section{Proof}
\subsection{Appropriate Output Domain}
\label{sec:approp-output-domain}
The raw type and domain are equivalent, save for potential nullity in the atomic type. 
The scalar scorer structurally cannot emit null.
Therefore the output of the function is a member of the output domain.

\subsection{Stability Guarantee}

The constructor first performs checks to ensure that the preconditions on \texttt{compute\_score} are met.
It checks that vectors in the input domain do not contain null values,
that the candidates are strictly increasing,
that alpha is fractional and in the range $[0, 1]$,
and computes a \texttt{size\_limit} for which $\texttt{size\_limit} \cdot \texttt{alpha\_den}$ does not overflow a \texttt{usize}.
Thus by the definition of \texttt{compute\_score}, for each candidate, the response from the function is:

\begin{equation}
    \textrm{compute\_score}(X, c, \alpha_{num}, \alpha_{den}, l) = |\alpha_{den} \cdot \min(\#(X < c), l), \alpha_{num} \cdot \min(|X| - \#(X = c), l)|
\end{equation}

The sensitivity of this function differs depending on if the size of the input vector is known.

\subsubsection{Unknown Size Stability}
\label{unsized-stability}
First, consider the case where the size is unknown.

\begin{lemma}
    \label{single-change-stab-sym}
    If $d_{Sym}(X, X') = 1$, then $d_\infty(\function(X), \function(X')) \le \max(\alpha_{num}, \alpha_{den} - \alpha_{num})$.
\end{lemma}

\begin{proof}
Assume $d_{Sym}(X, X') = 1$. 

\begin{align*}
    &d_\infty(\function(X)_i, \function(X')_i) \\
    &= \max\limits_{i} |\function(X)_i - \function(X')_i| &&\text{by definition of } d_\infty \\
    &= \max\limits_{i} |\mathrm{abs\_diff}(\alpha_{den} \cdot \min(\#(X < C_i), l), \alpha_{num} \cdot \min(|X| - \#(X = C_i), l)) &&\text{by definition of } \function
        \\&\hspace{1cm}\mathrm{abs\_diff}(\alpha_{den} \cdot \min(\#(X' < C_i), l), \alpha_{num} \cdot \min(|X'| - \#(X' = C_i), l))| \\
    &= \alpha_{den} \cdot \max\limits_{i} ||\min(\#(X < C_i), l) - \alpha \cdot \min(|X| - \#(X = C_i), l)|
        \\&\hspace{1cm}|\min(\#(X' < C_i), l) - \alpha \cdot \min(|X'| - \#(X' = C_i), l)|| \\
    &\le \alpha_{den} \cdot \max\limits_{i} ||\#(X < C_i) - \alpha \cdot (|X| - \#(X = C_i))|
        \\&\hspace{1cm}|\#(X' < C_i) - \alpha \cdot (|X| - \#(X' = C_i))|| \\
\intertext{Consider each of the three cases of adding or removing an element in X.}
\intertext{Case 1. Assume $X'$ is equal to $X$, but with some $X_j < C_i$ added or removed.}
    &= \alpha_{den} \cdot \max\limits_{i} ||(1 - \alpha) \cdot \#(X < C_i) - \alpha \cdot \#(X > C_i)| 
        \\&\hspace{1cm} - |(1 - \alpha) \cdot (\#(X < C_i) \pm 1) - \alpha \cdot \#(X > C_i)|| \\
    &\leq \alpha_{den} \cdot  \max\limits_{i} ||(1 - \alpha) \cdot \#(X < C_i) - \alpha \cdot \#(X > C_i)|
        \\&\hspace{1cm} - (|(1 - \alpha) \cdot \#(X < C_i) - \alpha \cdot \#(X > C_i)| + |\pm(1 - \alpha)|)| &&\text{by triangle inequality} \\
    &= \alpha_{den} \cdot \max\limits_{i} |1 - \alpha| &&\text{scores cancel}\\
    &= \alpha_{den} - \alpha_{num} &&\text{since } \alpha \leq 1
\intertext{Case 2. Assume $X'$ is equal to $X$, but with some $X_j > C_i$ added or removed.}
    &= \alpha_{den} \cdot \max\limits_{i} ||(1 - \alpha) \cdot \#(X < C_i) - \alpha \cdot \#(X > C_i)| 
        \\&\hspace{1cm} - |(1 - \alpha) \cdot \#(X < C_i) - \alpha \cdot (\#(X > C_i) \pm 1)|| \\
    &\leq \alpha_{den} \cdot \max\limits_{i} ||(1 - \alpha) \cdot \#(X < C_i) - \alpha \cdot \#(X > C_i)| 
        \\&\hspace{1cm} - (|(1 - \alpha) \cdot \#(X < C_i) - \alpha \cdot \#(X > C_i)| + |\pm\alpha|)| &&\text{by triangle inequality} \\
    &= \alpha_{den} \cdot \max\limits_{i} |\alpha| &&\text{scores cancel}\\
    &= \alpha_{num} &&\text{since } \alpha \geq 0
\intertext{Case 3. Assume $X'$ is equal to $X$, but with some $X_j = C_i$ added or removed.}
    &= \alpha_{den} \cdot \max\limits_{i} ||(1 - \alpha) \cdot \#(X < C_i) - \alpha \cdot \#(X > C_i)| 
        \\&\hspace{1cm} - |(1 - \alpha) \cdot \#(X < C_i) - \alpha \cdot \#(X > C_i)|| \\
    &= 0 &&\text{no change in score}
\intertext{Take the union bound over all cases.}
    &\le \max(\alpha_{num}, \alpha_{den} - \alpha_{num})
\end{align*}

\end{proof}

Take any two elements $X, X'$ in the \texttt{input\_domain} and any pair $(\din, \dout)$, 
where \din\ has the associated type for \texttt{input\_metric} and \dout\ has the associated type for \texttt{output\_metric}.
Assume $X, X'$ are \din-close under \texttt{input\_metric} and that $\texttt{stability\_map}(\din) \leq \dout$. 

\begin{align*}
    \dout &= \max\limits_{X \sim X'} d_{\infty}(s, s') &&\text{where } s = \function(X)\\
    &= \max\limits_{X \sim X'} \max\limits_{i} |s_i - s'_i| &&\text{by definition of }\rustdoc{metrics/struct}{LInfDistance} \text{, without monotonicity} \\
    &\leq \sum_j^{\din} \max\limits_{Z_{j} \sim Z_{j+1}} \max\limits_{i} |s_{i,j} - s_{i,j+1}| &&\text{by path property } d_{Sym}(Z_i, Z_{i+1}) = 1, X = Z_0 \text{ and } X' = Z_\din \\
    &\leq \sum_j^{\din} \max(\alpha_{num}, \alpha_{den} - \alpha_{num}) &&\text{by \ref{single-change-stab-sym}} \\
    &\leq \din \cdot \max(\alpha_{num}, \alpha_{den} - \alpha_{num}) \\
\end{align*}
\label{sec:unknown-size}
This formula matches the stability map in the case where the dataset size is unknown.


\subsubsection{Known Size Stability}
\label{sec:known-size}
Now consider the case where the dataset size is known.

\begin{lemma}
    \label{single-change-stab}
    If $d_{CO}(X, X') \le 1$, then $d_\infty(\function(X), \function(X')) \le 2 \cdot \alpha_{den}$.
\end{lemma}

\begin{proof}
Assume $d_{CO}(X, X') \le 1$. 

\begin{align*}
    &d_\infty(\function(X), \function(X')) \\
    &= \max\limits_{i} |\function(X)_i - \function(X')_i| &&\text{by definition of } d_\infty \\
    &= \max\limits_{i} |\mathrm{abs\_diff}(\alpha_{den} \cdot \min(\#(X < C_i), l), \alpha_{num} \cdot \min(|X| - \#(X = C_i), l)) &&\text{by def. of } \function
        \\&\hspace{1cm} -\mathrm{abs\_diff}(\alpha_{den} \cdot \min(\#(X' < C_i), l), \alpha_{num} \cdot \min(|X'| - \#(X' = C_i), l))| \\
    &= \alpha_{den} \cdot \max\limits_{i} ||\min(\#(X < C_i), l) - \alpha \cdot \min(|X| - \#(X = C_i), l)|
        \\&\hspace{1cm}-|\min(\#(X' < C_i), l) - \alpha \cdot \min(|X'| - \#(X' = C_i), l)|| \\
    &= \alpha_{den} \cdot \max\limits_{i} ||\#(X < C_i) - \alpha \cdot (|X| - \#(X = C_i))|
        \\&\hspace{1cm}-|\#(X' < C_i) - \alpha \cdot (|X| - \#(X' = C_i))|| \\
\end{align*}

\begin{align*}
\intertext{Consider each of the four cases of changing a row in X.}
\intertext{Case 1. Assume $X'$ is equal to $X$, but with some $X_j < C_i$ replaced with $X'_j > C_i$.}
    &= 2 \cdot \alpha_{den} \cdot \max\limits_{i} ||(1 - \alpha) \cdot \#(X < C_i) - \alpha \cdot \#(X > C_i)| 
        \\&\hspace{1cm} - (1 - \alpha) \cdot (\#(X < C_i) - 1) - \alpha \cdot (\#(X > C_i) + 1)|| &&\text{by definition of }\function \\
    &\leq 2 \cdot \alpha_{den} \cdot \max\limits_{i} ||(1 - \alpha) \cdot \#(X < C_i) - \alpha \cdot \#(X > C_i)|
        \\&\hspace{1cm} - (|(1 - \alpha) \cdot \#(X < C_i) - \alpha \cdot \#(X > C_i)| + |1|)| &&\text{by triangle inequality} \\
    &= 2 \cdot \alpha_{den} \cdot \max\limits_{i} |1| &&\text{scores cancel}\\
    &= 2 \cdot \alpha_{den}
\intertext{Case 2. Assume $X'$ is equal to $X$, but with some $X_j > C_i$ replaced with $X'_j < C_i$.}
    &= 2 \cdot \alpha_{den}
    \intertext{\hspace{1cm}by symmetry, follows from Case 1.}
\intertext{Case 3. Assume $X'$ is equal to $X$, but with some $X_j \neq C_i$ replaced with $C_i$.}
    &\leq 2 \cdot \max(\alpha_{num}, \alpha_{den} - \alpha_{num})\\
    \intertext{\hspace{1cm}equivalent to one removal (see \rustdoc{transformations/fn}{make\_quantile\_score\_candidates})}
\intertext{Case 4. Assume $X'$ is equal to $X$, but with some $X_j = C_i$ replaced with $X'_j \neq C_i$.}
    &\leq 2 \cdot \max(\alpha_{num}, \alpha_{den} - \alpha_{num}) \\
\intertext{\hspace{1cm}equivalent to one addition (see \rustdoc{transformations/fn}{make\_quantile\_score\_candidates})}
\intertext{Take the union bound over all cases.}
    d_\infty(s_i, s'_i) &\le \max(2 \cdot \alpha_{den}, 2 \cdot \max(\alpha_{num}, \alpha_{den} - \alpha_{num})) = 2 \cdot \alpha_{den}
    \intertext{\hspace{1cm}since $\max(\alpha, 1 - \alpha) \leq 1$}
\end{align*}
\end{proof}

Take any two elements $X, X'$ in the \texttt{input\_domain} and any pair $(\din, \dout)$, 
where \din\ has the associated type for \texttt{input\_metric} and \dout\ has the associated type for \texttt{output\_metric}.
Assume $X, X'$ are \din-close under \texttt{input\_metric} and that $\texttt{stability\_map}(\din) \leq \dout$. 

\begin{align*}
    \dout &= \max\limits_{X \sim X'} d_{\infty}(s, s') \\
    &= \max\limits_{X \sim X'} \max\limits_{i} |s_i - s'_i| &&\text{by definition of }\rustdoc{metrics/struct}{LInfDistance}\text{, without monotonicity} \\
    &\leq \sum_j^{\din // 2} \max\limits_{Z_{j} \sim Z_{j+1}} \max\limits_{i} |s_{i,j} - s_{i,j+1}| &&\text{by path property } d_{CO}(Z_i, Z_{i+1}) = 1, X = Z_0 \text{ and } Z_\din = X' \\
    &\leq \sum_j^{\din // 2} 2 \cdot \alpha_{den} &&\text{by \ref{single-change-stab}} \\
    &\leq 2 \cdot (\din // 2) \cdot \alpha_{den} \\
\end{align*}

This formula matches the stability map in the case where the dataset size is known.

\subsubsection{Conclusion}
Take any two elements $X, X'$ in the \texttt{input\_domain} and any pair $(\din, \dout)$, 
where \din\ has the associated type for \texttt{input\_metric} and \dout\ has the associated type for \texttt{output\_metric}.
Assume $X, X'$ are \din-close under \texttt{input\_metric} and that $\texttt{stability\_map}(\din) \leq \dout$. 

By \ref{sec:unknown-size} and \ref{sec:known-size} it is shown that \function(X), \function(X') are \dout-close under \texttt{output\_metric} for any choice of input arguments.


\bibliographystyle{plain}
\bibliography{references.bib}

\end{document}
